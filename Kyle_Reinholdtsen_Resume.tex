\documentclass[11pt,a4paper,sans]{moderncv}
\usepackage[scale=0.9, bottom=2.5cm]{geometry}
\usepackage{paralist}
\usepackage{lmodern}
\usepackage[colorlinks=true, urlcolor=blue]{hyperref}

\moderncvcolor{blue}
\moderncvstyle{casual}

\name{Kyle}{Reinholdtsen}
\title{Software Engineer}  
\mobile{(425)~215-2474}
\address{Greater Seattle Area}{WA}{}
\email{kyle@reinholdtsen.com}
\social[linkedin][linkedin.com/in/kyle-reinholdtsen-306849223/]{Kyle Reinholdtsen}
\homepage{kyle.reinholdtsen.com}

\begin{document}

\makecvtitle

\vspace{-1.5em}

\section{About}

\vspace{-1.25em}
\cventry{}{}{}{}{}{
  Full-stack software engineer with hands-on experience building production-ready web apps using React, Angular, Node.js, and AWS. Passionate about frontend UX, scalable backend design, and cloud-native deployments using Docker and serverless architecture. Strong foundation in distributed systems and team-based software delivery.
}

\section{Skills}

\vspace{-1.25em}
\cventry{}{}{}{}{}{
  \begin{compactitem}
    \item \textbf{Languages:} TypeScript, JavaScript, Python, Java, HTML, CSS
    \item \textbf{Frontend:} Angular, React, Redux, Responsive Design, RxJS
    \item \textbf{Backend:} Node.js, Express, REST APIs
    \item \textbf{AWS:} S3, Lambda, DynamoDB, Fargate, ECS, API Gateway
    \item \textbf{DevOps \& Tools:} Git, Docker, CI/CD, Linux
    \item \textbf{Certifications:} AWS Certified Cloud Practitioner
  \end{compactitem}
}

\section{Technical Experience}

\cventry{01/2022--03/2023}{Software Engineer}{Husky Robotics, University of Washington}{Seattle}{}{
  \begin{compactitem}
    \item Built the mission control website for Husky Robotics, a student engineering team, to operate the team’s rover using JavaScript, React, and Redux.
    \item Participated in peer code reviews, wrote and reviewed GitHub pull requests, and maintained high code quality through feedback cycles and issue tracking.
    \item Used Git for version control, contributed to feature branches, and helped enforce coding standards and write documentation.
    \item Designed and implemented a custom WebSocket-based JSON messaging protocol to enable real-time, bidirectional communication between the mission control website and the rover.
    \item Created UI elements to display rover camera feeds and telemetry data (position, power, velocity).
    \item Implemented a 3D rendering of the rover with React Three Fiber, dynamically updated in real-time using telemetry data.
  \end{compactitem}
}

\section{Projects}

\cventry{}{Paintle}{\href{https://paintle.net}{\textnormal{paintle.net}}}{\href{https://github.com/KyleR56/paintle-front-end}{\textnormal{GitHub}}}{}{
  \begin{compactitem}
    \item Created Paintle, a website inspired by Wordle where users solve a daily puzzle by painting a 5x5 grid.
    \item Built a responsive frontend using Angular, designed for seamless play on both desktop and mobile browsers.
    \item Deployed a fully serverless architecture on AWS using S3 (frontend), Lambda (backend logic), API Gateway (routing), and DynamoDB (storage), achieving <$200\,$ms cold-start latency and \$0 backend cost under typical usage.
    \item Implemented secure authentication via Google OAuth 2.0 and JSON Web Tokens (JWT), incorporating best practices such as short-lived tokens, RS256 signing, and validation of token audience and issuer.
  \end{compactitem}
}

\cventry{}{MCQuest}{\href{https://github.com/KyleR56/mc-quest}{\textnormal{GitHub}}}{}{}{
  \begin{compactitem}
    \item Developed a custom Minecraft server to support a multiplayer RPG with questing, combat, and character progression.
    \item Implemented backend systems in Java for handling in-game mechanics such as physics calculations, skill behavior, item interactions, and data persistence.
    \item Collaborated with a 5-person team using Agile practices including structured requirements planning, sprint-based development, weekly standups, and CI/CD pipelines to ensure quality and fast iteration.
  \end{compactitem}
}

\section{Education}

\cventry{06/2021--12/2023}{B.S. Computer Science \& Engineering}{University of Washington}{Seattle}{}{
  \begin{compactitem}
    \item \textbf{GPA:} 3.88 (Cum Laude)
  \end{compactitem}
}

\end{document}
