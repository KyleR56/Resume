\documentclass[11pt,a4paper,sans]{moderncv}
\usepackage[scale=0.9, bottom=2.5cm]{geometry}
\usepackage{paralist}
\usepackage{lmodern}
\usepackage[colorlinks=true, urlcolor=blue]{hyperref}

\moderncvcolor{blue}
\moderncvstyle{casual}

\name{Kyle}{Reinholdtsen}
\title{Software Engineer}  
\mobile{(425)~215-2474}
\address{Seattle}{WA}{}
\email{kyle@reinholdtsen.com}
\social[linkedin][linkedin.com/in/kyle-reinholdtsen-306849223/]{Kyle Reinholdtsen}
\homepage{kyle.reinholdtsen.com}

\begin{document}

\makecvtitle

\vspace{-1.5em}

\section{About}

\vspace{-1em}
\cventry{}{}{}{}{}{
  Proficient full stack developer with 1+ years of experience in web programming, object-oriented design, distributed systems, and machine learning. Thrives in collaborative environments and works effectively in teams to deliver high-quality software.
}

\section{Skills}

\cventry{}{Languages}{}{}{}{
  \begin{compactitem}
    \item Java, JavaScript, TypeScript, HTML, CSS, Python, C\#, SQL.
  \end{compactitem}
}

\cventry{}{Tools \& Platforms}{}{}{}{
  \begin{compactitem}
    \item React, Angular, Node.js, Express, REST APIs, HTTP, Docker, AWS, Linux, Git,.
  \end{compactitem}
}

\cventry{}{Certifications}{}{}{}{
  \begin{compactitem}
    \item AWS Certified Cloud Practitioner.
  \end{compactitem}
}

\section{Education}

\cventry{06/2021--12/2023}{B.S. Computer Science \& Engineering}{}{University of Washington}{Seattle}{
  \begin{compactitem}
    \item \textbf{\textit{Classes:}} Distributed Systems, Operating Systems, Algorithms, Compilers, Machine Learning, Artificial Intelligence, Software Engineering, Computer Vision, Programming Languages, Systems Programming, Database Systems, Interaction Programming.
    \item \textbf{\textit{GPA:}} 3.88 (Cum Laude, Dean's List).
  \end{compactitem}
}

\section{Experience}

\cventry{01/2022--03/2023}{Software Engineer}{Husky Robotics}{University of Washington}{Seattle}{
  \begin{compactitem}
    \item Developed the mission control website for operating the team’s rover using JavaScript, React, and
    Redux.
    \begin{compactitem}
      \item Designed and implemented a custom WebSocket-based messaging protocol to enable real-time, bidirectional communication between the mission control website and the rover.
      \item Created UI elements to display rover camera feeds and telemetry data (position, power, velocity).
      \item Implemented a 3D rendering of the rover with React Three Fiber, dynamically updated in real-time using telemetry data.
    \end{compactitem}
    \item Developed a Unity simulator that emulates the rover’s cameras, motors, wheels, and sensors in a 3D
    virtual environment, and transmits camera feeds and telemtry data to the mission control website.
    \begin{compactitem}
      \item Enabled users to orbit the rover within the simulator using intuitive mouse controls.
      \item Added functionality to configure the rover's intrinsic camera parameters for enhanced simulation accuracy.
    \end{compactitem}
  \end{compactitem}
}

\section{Projects}

\cventry{}{MCQuest}{}{}{}{
  \begin{compactitem}
    \item Collaborated with a team of 5 to create an online multiplayer game within Minecraft, where players explore a fantasy world, complete quests, slay enemies, and upgrade their character.
    \item Followed an Agile development cycle with continuous integration and continuous deployment (CI/CD) to streamline updates, automate testing, and maintain code quality.
    \item Developed server-side game logic using Java such as physics calculations, item and skill behavior, and
    data persistence.
  \end{compactitem}
}

\cventry{}{Paintle}{\href{https://paintle.net}{\textnormal{paintle.net}}}{\href{https://github.com/KyleR56/paintle-front-end}{\textnormal{GitHub}}}{}{
  \begin{compactitem}
    \item Created Paintle, a website inspired by Wordle where users solve a daily puzzle by painting a 5x5 grid.
    \item Developed the frontend using Angular, ensuring an accessible interface for both desktop and mobile users.
    \item Built the backend with Node.js and Express to serve webpage content and handle API requests.
    \item Deployed the application using Docker containers on AWS Fargate, leveraging load balancing for scalability.
  \end{compactitem}
}

\end{document}
